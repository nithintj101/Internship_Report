\section{Conclusion}
\label{sec:conc}

As the growing concerns about global $CO_2$ emissions are now at all time high. Zero Emission Fuels B.V aims to address these concerns by designing and fabricating a micro plant. This micro-plant captures water vapour and carbon-dioxide from the atmosphere using a direct air capture (DAC) unit. The captured carbon-dioxide is then reacted with hydrogen that is generated from the captured water vapor from the alkaline electrolysis cell (AEC) unit of the micro plant to produced grade AA methanol that can be used in industries or as synthetic fuel in a methanol synthesis (MS) reactor.   
\bigbreak
\noindent
This report mainly focuses on the fabrication of the DAC unit of the ZEF micro-plant. Before designing and fabricating the DAC system, the previous iterations of the DAC unit were studied and noted down in Section \ref{sec:prework}. Based on these works, the new system was designed and fabricated which met with its own issues. This led to the fabrication of the new DAC V2.0 system which rectified this issues of the first system. However, due to constraints the setup couldn't complete testing and couldn't be run before the internship period. The design, component and process changes are all elaborated further in the report.    

\bigbreak
\noindent During the course of my internship in ZEF B.V. I have learnt a lot about LT DAC systems that have been utilized in the market at the moment and the ongoing research in the field. I was also able to engineer two LT DAC systems (DAC V1.1 system and DAC V2.0 system). Since, it was a interdisciplinary project that had a bit of everything. I had the immense pleasure of learning things that I had no experience such as working with chemicals, designing in fusion360, 3D printing and arduino programming. It also gave me the opportunity to work with a multi-cultural group that helped me understand a lot about the various cultures as well. 
